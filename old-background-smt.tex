

\subsection{Background: SAT modulo Theory (SMT)}
\label{sec:background-smt}
%A Boolean Satifiability (SAT) problem is defined by a set of Boolean variables $v_1,\ldots v_n$ and a Boolean formula $\Phi$ defined over these variables. A solution to a SAT problem is an assignment to these variables such that $\Phi$ is satisfied, or UNSAT if there is no assignment of these variables that satisfies $\Phi$. 
%While the SAT problem is famously known to be NP-Complete, modern SAT solvers are extremely efficient, solving formulaes with millions of variables in reasonable time. Nevertheless, 

While modern SAT solvers are extremely powerful, the set of problems one can solve with SAT is inherently limited to decision problems that can be translated to Boolean formulaes. 
SAT modulo Theory (SMT)~\cite{DBLP:journals/jacm/NieuwenhuisOT06,DBLP:journals/constraints/BofillPSV12,DBLP:conf/cp/Nieuwenhuis10} is an approach designed to leverage the power of modern SAT solvers while applying them to a larger set of problems, as follows. 


% Define SMT
%\pavel{Suggest to use $DECIDE_{MAPF_R}$ instead of \decidet because $T=MAPF_R$.}
Let $\Gamma$ be a decision problem in some complex logic theory $T$. 
The basic use of SMT divides $\Gamma$ into two parts. The first, called the \ps, is an abstraction of $\Gamma$ that keeps only its Boolean structure. The second, called \decidet, is a decision procedure that accepts an assignment that satisfies the \ps and outputs \true if this assignment 
corresponds to a solution of the original problem $\Gamma$ that is valid with respect to axioms of the underlying theory $T$. 
If \decidet returns \false, i.e., if it is given a solution to the \ps that cannot be mapped to a solution for $\Gamma$, then \decidet returns also a \emph{conflict}  (often called a {\em lemma}) that explains why the solution to the \ps is not valid. 


%hope it is Ok I changed to conflict, to be consistent
% How to solve SMT
The standard \smt solving procedure is iterative. First, find a satisfying assignment of the \ps. 
Then, call \decidet with this assignment. If \decidet returns \true, the satisfying assignment is returned and the \smt solving procedure is finished. 
Otherwise, the \ps is extended with new constraints designed to resolve the conflict returned by \decidet. 
In more general cases, not only new constraints are added to resolve a conflict but also new propositional variables.

%\roni{Pavel: I tried to re-write some of this text so that it is easier for a SMT-dumb reader like me. Please verify that this makes sense, and, of course, feel free to edit or revert back to your text.} \pavel{Makes sense, I also understand SMT better now :-) }

%The standard SAT-solving procedure then decides what variables should be assigned $\mathit{TRUE}$ in order to satisfy the skeleton - these variables tells what atoms hold in $\Gamma$. $\mathit{DECIDE_T}$ then checks if the conjunction of atoms assigned $\mathit{TRUE}$ is valid with respect to axioms of $T$. If so then satisfying assignment is returned and we are finished. Otherwise a conflict from $\mathit{DECIDE_T}$ (often called a {\em lemma}) is reported back to the SAT solver and the skeleton is extended with new constraints resolving the conflict. In more general cases, not only new constraints are added to resolve a conflict but also new variables i.e. atoms can be added to $\Gamma$.


%fragment of $T$ restricted on {\em conjunctive formulae}. \roni{I don't understand: what does it mean to decide a fragment of $T$?} 
%A general $T$-formula $\Gamma$ being decided for satisfiability is transformed to a {\em propositional skeleton} by replacing its atoms with propositional variables. The standard SAT-solving procedure then decides what variables should be assigned $\mathit{TRUE}$ in order to satisfy the skeleton - these variables tells what atoms hold in $\Gamma$. $\mathit{DECIDE_T}$ then checks if the conjunction of atoms assigned $\mathit{TRUE}$ is valid with respect to axioms of $T$. If so then satisfying assignment is returned and we are finished. Otherwise a conflict from $\mathit{DECIDE_T}$ (often called a {\em lemma}) is reported back to the SAT solver and the skeleton is extended with new constraints resolving the conflict. In more general cases, not only new constraints are added to resolve a conflict but also new variables i.e. atoms can be added to $\Gamma$.


To demonstrate SMT and the standard SMT solving procedure described above, consider a simple example, defined for the following formula:
\begin{equation}
\Gamma := ((a = b) \wedge (a=c)) \vee (\neg(b = c) \wedge \neg(a=d))
\end{equation}
The problem is which non-Boolean values to assign to $a,b,c,d$ such that the formula is true. 
Note that in any solution to $\Gamma$, the common axioms of equality, such as transitivity, must hold. 
Consider the \ps  
\begin{equation}
 (X_{ab} \wedge X_{ac}) \vee (\neg X_{bc} \wedge \neg X_{ad})   
\end{equation}
where $X_{ab}$, $X_{ac}$, $X_{bc}$, and $X_{ad}$ are propositional variables such that $x_{ij}$ represents that $i=j$ for $i,j=\{a,b,c\}$. 
That is, the propositional variables in this \ps correspond to the different clauses in $\Gamma$. 
The SAT solver can set $X_{ab}=true$, $X_{ac}=true$, $X_{bc}=false$, and $X_{ad}=false$ to satisfy this \ps. % which in original $\Gamma$ is interpreted as satisfying individual atoms according to the truth value assignment. 
The corresponding solution is inconsistent with the semantic of equality (EQ) theory, due to transitivity of equality. Namely, if $x_{ab}$ and $x_{ac}$ are both true, then $x_{ac}$ must also be true. Note that the SAT solver does not know this, as it is not represented in the \ps. A suitable DECIDE$_{EQ}$ is needed to check that equality theory holds and suggest a conflict otherwise. In our case, DECIDE$_{EQ}$ can suggest the conflict $X_{ab} \wedge X_{ac} \rightarrow X_{bc}$. This conflict is added as to \ps, so the SAT solver can give a new assignment. %}. 
In this way, knowledge from the underlying theory is propagated by the conflict to the \ps level.

